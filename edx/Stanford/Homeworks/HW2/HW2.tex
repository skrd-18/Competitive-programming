%CS 109 Problem Set Template

\title{CS109 PSET Template}
\documentclass{article}
	% basic article document class
	% use percent signs to make comments to yourself -- they will not show up.
\usepackage{amsmath}
\usepackage{amssymb}
	% packages that allow mathematical formatting
\usepackage{graphicx}
	% package that allows you to include graphics
    %includegraphic[width=\textwidth]{FILENAME}
\usepackage[top=1in, bottom=1in, left=1in, right=1in]{geometry}
\frenchspacing
	% one space after periods
\usepackage{fancyhdr}
	% allows custom headers
\usepackage{relsize}
\pagestyle{fancy}
\setlength{\headheight}{24pt}% ...at least 23.1004pt
\lhead{edX, Stanford University \\ Problem Set 4} 
\rhead{Shiva}
\cfoot{\thepage}
\renewcommand{\footrulewidth}{0.4pt} 
	%footer

\begin{document}
\thispagestyle{fancy} %shows header/footer
\begin{enumerate}

    % Problem 1 --------------------------------------------------------------
    \item Given an adjacency-list representation of a directed graph, where each vertex maintains an array of its outgoing edges (but *not* its incoming edges), how long does it take, in the worst case, to compute the in-degree of a given vertex? As usual, we use $n$ and $m$ to denote the number of vertices and edges, respectively, of the given graph. Also, let $k$ denote the maximum in-degree of a vertex. (Recall that the in-degree of a vertex is the number of edges that enter it.)

\end{enumerate}
\newpage
\end{document}
