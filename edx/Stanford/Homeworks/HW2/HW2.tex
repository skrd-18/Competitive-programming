\title{Homework 2}
\documentclass{article}
	% basic article document class
	% use percent signs to make comments to yourself -- they will not show up.
\usepackage{amsmath}
\usepackage{amssymb}
	% packages that allow mathematical formatting
\usepackage{graphicx}
	% package that allows you to include graphics
    %includegraphic[width=\textwidth]{FILENAME}
\usepackage[top=1in, bottom=1in, left=1in, right=1in]{geometry}
\frenchspacing
	% one space after periods
\usepackage{fancyhdr}
	% allows custom headers
\usepackage{relsize}
\pagestyle{fancy}
\setlength{\headheight}{24pt}% ...at least 23.1004pt
\lhead{edX, Stanford University \\ Problem Set 2} 
\rhead{Shiva}
\cfoot{\thepage}
\renewcommand{\footrulewidth}{0.4pt} 
	%footer

\begin{document}
\thispagestyle{fancy} %shows header/footer

\section*{Part I: Key Concepts Revision}

Before diving into the specific problems, here is a quick summary of the core concepts required to solve them.

\subsection*{1. The Master Theorem}
Used to solve recurrences of the form $T(n) = aT(n/b) + f(n)$, where $a \ge 1$, $b > 1$. $a$ is the number of subproblems, $n/b$ is the size of each subproblem, and $f(n)$ is the cost of the work done outside the recursive calls.
Compare $f(n)$ with the critical exponent $n^{\log_b a}$:
\begin{itemize}
	\item \textbf{Case 1 (Leaf Heavy):} If $f(n) = O(n^{\log_b a - \epsilon})$ for $\epsilon > 0$, then $T(n) = \Theta(n^{\log_b a})$.
	\item \textbf{Case 2 (Balanced):} If $f(n) = \Theta(n^{\log_b a})$, then $T(n) = \Theta(n^{\log_b a} \log n)$.
	\item \textbf{Case 3 (Root Heavy):} If $f(n) = \Omega(n^{\log_b a + \epsilon})$ for $\epsilon > 0$ and satisfies the regularity condition, then $T(n) = \Theta(f(n))$.
\end{itemize}

\subsection*{2. QuickSort Analysis}
\begin{itemize}
	\item \textbf{Pivot Quality:} The performance depends heavily on the split balance.
	\item \textbf{Best Case:} Perfectly balanced splits ($n/2$ vs $n/2$). Depth is $\Theta(\log n)$.
	\item \textbf{Worst Case:} Highly unbalanced splits ($0$ vs $n-1$). Depth is $\Theta(n)$.
	\item \textbf{Probability:} A random pivot is "good" (yielding a split better than $\alpha$ to $1-\alpha$) with probability $1 - 2\alpha$.
\end{itemize}

\subsection*{3. Recursion Tree Depth}
For a problem of size $n$ that shrinks by a factor of $X$ at each step (e.g., $n \to n/X$):
\begin{itemize}
	\item The depth of the recursion tree is $\log_X n$.
	\item Using change of base formula: $\log_X n = \frac{\ln n}{\ln X}$.
\end{itemize}

\newpage

\section*{Part II: Problem Solutions}

\subsection*{Question 1: Master Method Application}
\textbf{Problem:} Solve $T(n) = 5T(n/3) + 4n$.

\textbf{Solution:}
\begin{enumerate}
	\item Identify parameters: $a = 5, b = 3, f(n) = 4n$.
	\item Calculate the critical exponent: $\log_b a = \log_3 5$.
	\item Estimate the value: Since $3^1=3$ and $3^2=9$, we know $1 < \log_3 5 < 2$. Specifically, $\log_3 5 \approx 1.46$.
	\item Compare $f(n)$ with $n^{\log_3 5}$:
	      \[ f(n) = \Theta(n^1) \quad \text{vs} \quad n^{1.46} \]
	      Since $1 < 1.46$, $f(n)$ grows polynomially slower than $n^{\log_3 5}$. This fits \textbf{Case 1}.
	\item \textbf{Result:} $T(n) = \Theta(n^{\log_3 5})$.
\end{enumerate}

\subsection*{Question 2: FastPower Algorithm}
\textbf{Problem:} Analyze the running time of the recursive exponentiation algorithm.

\textbf{Solution:}
\begin{itemize}
	\item The algorithm computes $a^b$.
	\item \textbf{Recursive Step:} It computes $a^{\lfloor b/2 \rfloor}$ recursively.
	\item \textbf{Work per step:} Constant time arithmetic operations ($\Theta(1)$).
	\item \textbf{Recurrence:} $T(b) = T(b/2) + \Theta(1)$.
	\item This recurrence describes a process where the input is halved at every step. The depth is logarithmic.
	\item \textbf{Result:} $\Theta(\log b)$.
\end{itemize}

\subsection*{Question 3: QuickSort Pivot Probability}
\textbf{Problem:} Probability that a random pivot produces a split where the smaller side is $\ge \alpha n$.

\textbf{Solution:}
\begin{enumerate}
	\item Consider the sorted order of elements $1$ to $n$.
	\item To have a "bad" split (smaller side $< \alpha n$), the pivot must be in the first $\alpha n$ elements (too small) or the last $\alpha n$ elements (too big).
	\item Total "bad" pivots = $\alpha n + \alpha n = 2\alpha n$.
	\item Total "good" pivots = Total elements - Bad pivots = $n - 2\alpha n$.
	\item Probability = $\frac{\text{Good Pivots}}{\text{Total Pivots}} = \frac{n(1 - 2\alpha)}{n}$.
	\item \textbf{Result:} $1 - 2\alpha$.
\end{enumerate}

\subsection*{Question 4: Recursion Depth Range}
\textbf{Problem:} Find the range of depth $d$ if split sizes are between $\alpha k$ and $(1-\alpha)k$.

\textbf{Solution:}
The depth $d$ is determined by how fast $n$ reduces to 1.
\begin{itemize}
	\item \textbf{Minimum Depth (Fastest Reduction):}
	      Occurs when we always get the smallest fraction $\alpha$.
	      \[ n \cdot \alpha^d = 1 \implies d = \log_{1/\alpha} n = -\frac{\log n}{\log \alpha} \]

	\item \textbf{Maximum Depth (Slowest Reduction):}
	      Occurs when we always get the largest fraction $(1-\alpha)$.
	      \[ n \cdot (1-\alpha)^d = 1 \implies d = \log_{1/(1-\alpha)} n = -\frac{\log n}{\log (1-\alpha)} \]
\end{itemize}
\textbf{Result:} $-\frac{\log n}{\log \alpha} \le d \le -\frac{\log n}{\log (1-\alpha)}$

\subsection*{Question 5: QuickSort Recursion Depth (Min/Max)}
\textbf{Problem:} Minimum and maximum recursion depth of QuickSort.

\textbf{Solution:}
\begin{itemize}
	\item \textbf{Minimum Depth (Best Case):} The pivot is always the median, splitting the array into $n/2$ and $n/2$. The tree is perfectly balanced.
	      \[ \text{Depth} = \Theta(\log n) \]
	\item \textbf{Maximum Depth (Worst Case):} The pivot is always the min or max, splitting the array into $0$ and $n-1$. The tree becomes a linked list.
	      \[ \text{Depth} = \Theta(n) \]
\end{itemize}
\textbf{Result:} Min: $\Theta(\log n)$, Max: $\Theta(n)$.

\section*{Part III: Optional Theory Problems}

\subsection*{1. Square Root Recurrence}
\textbf{Recurrence:} $T(n) \le T(\lfloor \sqrt{n} \rfloor) + 1$.
\textbf{Solution:}
Let $n = 2^k$. Then $\sqrt{n} = 2^{k/2}$. The recurrence on the exponent $k$ is $S(k) = S(k/2) + 1$.
This means the exponent is halved at every step. It takes $\log k$ steps to reach base case.
Since $k = \log n$, the result is $\Theta(\log k) = \Theta(\log \log n)$.

\subsection*{2. Local Minimum in Grid}
\textbf{Problem:} Find local minimum in $n \times n$ grid in $O(n)$.
\textbf{Solution:}
Use Divide and Conquer.
\begin{enumerate}
	\item Find the minimum element on the central row and central column. Let this be $m$.
	\item If $m$ is a local minimum, return it.
	\item If not, move to the quadrant containing the neighbor smaller than $m$.
	\item Recurrence: $T(n) = T(n/2) + O(n)$ (where $O(n)$ is scanning the cross).
	\item This sums to $O(n)$.
\end{enumerate}

\newpage
\end{document}
