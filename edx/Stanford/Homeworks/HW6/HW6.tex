\documentclass[11pt]{article}
\usepackage[utf8]{inputenc}
\usepackage{geometry}
\usepackage{amsmath, amssymb, amsthm}
\usepackage{booktabs} % For nicer tables
\usepackage{xcolor}
\usepackage{hyperref}

% Page margins
\geometry{a4paper, margin=1in}

% Custom commands for math
\newcommand{\Prob}{\text{Pr}}
\newcommand{\Expect}{\mathbb{E}}
\newcommand{\modn}{\pmod{n}}

\title{\textbf{Comprehensive Study Guide: Hashing \& Bloom Filters}}
\author{edX Course - Stanford University}
\date{}

\begin{document}

\maketitle

\section{The Motivation for Hashing}
To maintain a set $S$ of keys drawn from a large universe $U$, we compare three fundamental data structures. The goal of hashing is to combine the speed of arrays with the space efficiency of linked lists.

\begin{table}[h]
    \centering
    \renewcommand{\arraystretch}{1.5}
    \begin{tabular}{@{}llll@{}}
        \toprule
        \textbf{Approach} & \textbf{Space} & \textbf{Time (Lookup)} & \textbf{Trade-off} \\ 
        \midrule
        \textbf{Direct Addressing} & $\Theta(|U|)$ & $O(1)$ Worst Case & \textbf{Fast but Impossible Space.} \\
        (Array) & & & If $U$ is large (e.g., IPv6), memory explodes. \\
        \midrule
        \textbf{Linked List} & $\Theta(|S|)$ & $\Theta(|S|)$ Worst Case & \textbf{Space Efficient but Slow.} \\
        & & & Must scan the entire list to find items. \\
        \midrule
        \textbf{Hash Table} & $\Theta(|S|)$ & $O(1)$ Expected & \textbf{The Sweet Spot.} \\
        & & & Maps huge $U$ to small table size $n$. \\
        \bottomrule
    \end{tabular}
    \caption{Comparison of Data Structures}
\end{table}

\section{Birthday Paradox}
Consider $n$ people with random birthdays (assuming 365 days in a year, equally likely, and independent). We want to find the smallest integer $n$ such that the probability of at least two people sharing a birthday is at least 50\%:
\begin{equation}
    P(\text{at least one collision}) \ge 0.5
\end{equation}

\subsection{The Complement Strategy}
Calculating collisions directly is complex due to the number of possible pairs. It is easier to calculate the complement event: \textbf{all birthdays are unique}.
\begin{align*}
    P(\text{shared}) &= 1 - P(\text{all unique}) \\
    1 - P(\text{unique}) &\ge 0.5 \\
    P(\text{unique}) &\le 0.5
\end{align*}

\subsection{Exact Probability Formula}
We assign birthdays to $n$ people one by one, ensuring no collisions:
\begin{itemize}
    \item Person 1: $365/365$ (Any day is fine)
    \item Person 2: $364/365$ (Must avoid Person 1)
    \item Person $k$: $(365 - (k-1))/365$ (Must avoid previous $k-1$ people)
\end{itemize}

The total probability is the product of these independent choices:
\begin{equation}
    P(\text{unique}) = 1 \times \frac{364}{365} \times \frac{363}{365} \times \dots \times \frac{365-(n-1)}{365}
\end{equation}

This can be written compactly using product notation:
\begin{equation}
    P(\text{unique}) = \prod_{k=0}^{n-1} \left( \frac{365-k}{365} \right) = \prod_{k=0}^{n-1} \left( 1 - \frac{k}{365} \right)
\end{equation}

\subsection{Approximation using Taylor Series}
To solve for $n$, we convert the product into a sum by taking the natural logarithm ($\ln$) of both sides:
\begin{equation}
    \ln(P(\text{unique})) = \sum_{k=0}^{n-1} \ln\left(1 - \frac{k}{365}\right)
\end{equation}

We use the Taylor Series expansion for $\ln(1-x)$ near $x=0$:
\[ \ln(1-x) = -x - \frac{x^2}{2} - \frac{x^3}{3} - \dots \]
For small $x$ (where $x = k/365$), we approximate $\ln(1-x) \approx -x$.

Substituting this into our sum:
\begin{align*}
    \ln(P(\text{unique})) &\approx \sum_{k=0}^{n-1} \left( -\frac{k}{365} \right) \\
    &= -\frac{1}{365} \sum_{k=0}^{n-1} k
\end{align*}

Using the arithmetic series sum formula $\sum_{k=0}^{n-1} k = \frac{n(n-1)}{2}$, we get:
\begin{equation}
    \ln(P(\text{unique})) \approx -\frac{1}{365} \cdot \frac{n(n-1)}{2} = -\frac{n(n-1)}{730}
\end{equation}

Exponentiating both sides gives the approximate probability formula:
\begin{equation}
    P(\text{unique}) \approx e^{-\frac{n(n-1)}{730}}
\end{equation}

\subsection{Solving for $n$}
We require $P(\text{unique}) \le 0.5$.
\begin{align*}
    e^{-\frac{n(n-1)}{730}} &\le 0.5 \\
    -\frac{n(n-1)}{730} &\le \ln(0.5) \\
    -\frac{n(n-1)}{730} &\le -0.693 \quad (\text{since } \ln(2) \approx 0.693) \\
    n(n-1) &\ge 0.693 \times 730 \\
    n(n-1) &\ge 505.89
\end{align*}

Approximating $n(n-1) \approx n^2$:
\begin{align*}
    n^2 &\approx 506 \\
    n &\approx \sqrt{506} \approx 22.49
\end{align*}

Since $n$ must be an integer, we round up:
\begin{equation}
    \boxed{n = 23}
\end{equation}

\section{Modular Arithmetic \& Inverses}

Understanding modular inverses is a prerequisite for the Universal Hashing proof.

\subsection{Modular Inverses}
An integer $u$ is said to be \textbf{invertible modulo $n$} if there exists an integer $v$ (often written as $u^{-1}$) such that:
\begin{equation}
    u \cdot v \equiv 1 \pmod{n}
\end{equation}

\begin{itemize}
    \item \textbf{Existence Condition:} An inverse exists \textbf{if and only if} $\gcd(u, n) = 1$.
    \item \textbf{Prime Modulus Property:} If $n = p$ (a prime number), then every non-zero integer $u \in \{1, \dots, p-1\}$ is invertible. This is because a prime number shares no factors with numbers smaller than itself.
\end{itemize}

\subsection{Computing Inverses: Extended Euclidean Algorithm (EEA)}
To find the inverse of $u \modn$, we solve Bézout's Identity: $s \cdot u + t \cdot n = 1$. Reducing this modulo $n$ gives $s \cdot u \equiv 1 \modn$.

\subsubsection*{Example: Find $11^{-1} \pmod{26}$}
\textbf{Step 1: Forward Pass (Euclidean Division)}
\begin{align*}
    26 &= 2 \cdot 11 + 4 \\
    11 &= 2 \cdot 4 + 3 \\
    4 &= 1 \cdot 3 + 1 \quad \leftarrow \text{GCD is 1, so inverse exists.}
\end{align*}

\textbf{Step 2: Backward Pass (Substitution)}
Express 1 as a linear combination of 11 and 26.
\begin{align*}
    1 &= 4 - 1 \cdot 3 \\
    &= 4 - 1 \cdot (11 - 2 \cdot 4) \quad \text{(Substitute } 3 = 11 - 2 \cdot 4 \text{)} \\
    &= 3 \cdot 4 - 1 \cdot 11 \\
    &= 3 \cdot (26 - 2 \cdot 11) - 1 \cdot 11 \quad \text{(Substitute } 4 = 26 - 2 \cdot 11 \text{)} \\
    &= 3 \cdot 26 - 6 \cdot 11 - 1 \cdot 11 \\
    &= 3 \cdot 26 - 7 \cdot 11
\end{align*}

\textbf{Step 3: Result}
Reducing modulo 26, the term $3 \cdot 26$ becomes 0:
\begin{align*}
    -7 \cdot 11 &\equiv 1 \pmod{26} \\
    -7 &\equiv 19 \pmod{26}
\end{align*}
\textbf{Conclusion:} $11^{-1} \equiv 19 \pmod{26}$.

\newpage

\section{Universal Hashing}

\subsection{Definition}
A family of hash functions $\mathcal{H}$ mapping $U \to \{0, \dots, n-1\}$ is \textbf{universal} if for any two distinct keys $x \neq y$:
\begin{equation}
    \Prob_{h \in \mathcal{H}}[h(x) = h(y)] \le \frac{1}{n}
\end{equation}
\textit{Note:} Per-key uniformity (probability $1/n$ of hitting a specific bucket) is necessary but \textbf{not sufficient}. We specifically require the \textbf{pairwise collision probability} to be low.

\subsection{Proof: Hashing IP Addresses}
\textbf{The Setup:}
\begin{itemize}
    \item \textbf{Keys:} IP addresses decomposed into 4 parts: $x = (x_1, x_2, x_3, x_4)$ where $x_i \in \{0, \dots, n-1\}$.
    \item \textbf{Buckets:} A prime number $n$.
    \item \textbf{Hash Family:} Defined by a random vector $a = (a_1, a_2, a_3, a_4)$ where each $a_i$ is uniform in $\{0, \dots, n-1\}$.
    \item \textbf{Function:} $h_a(x) = \sum_{i=1}^{4} a_i x_i \pmod{n}$.
\end{itemize}

\textbf{The Proof:}
We wish to calculate the probability that distinct keys $x$ and $y$ collide.

\textbf{1. The Collision Equation}
\begin{align*}
    h_a(x) &= h_a(y) \\
    \sum_{i=1}^{4} a_i x_i &\equiv \sum_{i=1}^{4} a_i y_i \pmod{n} \\
    \sum_{i=1}^{4} a_i (x_i - y_i) &\equiv 0 \pmod{n}
\end{align*}

\textbf{2. Isolate a Non-Zero Difference}
Since $x \neq y$, they must differ in at least one position. Without loss of generality, assume $x_4 \neq y_4$. We isolate the term involving $a_4$:
\begin{align*}
    a_4(x_4 - y_4) + \sum_{i=1}^{3} a_i (x_i - y_i) &\equiv 0 \pmod{n} \\
    a_4(x_4 - y_4) &\equiv - \sum_{i=1}^{3} a_i (x_i - y_i) \pmod{n}
\end{align*}
Let the right-hand side be $C = \sum_{i=1}^{3} a_i (y_i - x_i)$. The equation becomes:
\begin{equation}
    a_4(x_4 - y_4) \equiv C \pmod{n}
\end{equation}

\textbf{3. Principle of Deferred Decisions}
Imagine we pick the random coefficients $a_1, a_2, a_3$ \textbf{first}. This fixes the value of $C$. Now, we pick $a_4$.

Since $n$ is prime and $(x_4 - y_4) \not\equiv 0 \pmod{n}$, the term $(x_4 - y_4)$ has a unique modular inverse. We multiply both sides by this inverse:
\begin{equation}
    a_4 \equiv C \cdot (x_4 - y_4)^{-1} \pmod{n}
\end{equation}

\textbf{4. Conclusion}
The equation above yields exactly \textbf{one} valid solution for $a_4$.
Since $a_4$ is chosen uniformly from $n$ possibilities $\{0, \dots, n-1\}$, the probability of picking exactly this one solution is:
\begin{equation}
    \Prob[\text{Collision}] = \frac{1}{n}
\end{equation}
Thus, the family is Universal.

\newpage

\section{Bloom Filters}

A Bloom Filter is a space-efficient probabilistic data structure for set membership. It yields no false negatives, but possible false positives($h_i(x)$'s already set to 1 by other insertions.).

\subsection{Mathematical Derivation of False Positive Rate}
\textbf{Parameters:}
\begin{itemize}
    \item $n$: Number of items inserted.
    \item $m$: Size of the bit array.
    \item $k$: Number of hash functions.
\end{itemize}

\textbf{Step 1: Probability a bit is 0}
When inserting 1 element with 1 hash function, the probability a specific bit is \textit{not} set (remains 0) is:
$$ 1 - \frac{1}{m} $$
After inserting $n$ elements using $k$ hash functions, we perform $kn$ total writes. The probability the bit is still 0 is:
$$ \left(1 - \frac{1}{m}\right)^{kn} $$

\textbf{Step 2: Taylor Series Approximation}
Recall the Taylor series expansion for $e^{-x}$ for small $x$: $e^{-x} \approx 1 - x$.
Conversely, for large $m$:
$$ 1 - \frac{1}{m} \approx e^{-1/m} $$
Substituting this into our equation:
$$ \Prob[\text{bit is 0}] \approx \left( e^{-1/m} \right)^{kn} = e^{-kn/m} $$

\textbf{Step 3: Probability a bit is 1}
$$ \Prob[\text{bit is 1}] = 1 - \Prob[\text{bit is 0}] = 1 - e^{-kn/m} $$

\textbf{Step 4: False Positive Probability ($\epsilon$)}
A false positive occurs when we query an item that is \textbf{not} in the set, but all $k$ hash functions map to bits that happen to be 1. Assuming independence:
\begin{equation}
    \epsilon = \left( 1 - e^{-kn/m} \right)^k
\end{equation}

\section{Summary Checklist}
\begin{itemize}
    \item \textbf{Why Prime Buckets?} Prime $n$ ensures that every non-zero difference $(x_i - y_i)$ has a modular inverse. This guarantees the collision equation has a unique solution.
    \item \textbf{Invertible Slopes:} In affine hashing $h(x) = (ax+b) \pmod{m}$, if $a$ is not coprime to $m$ ($\gcd(a,m) > 1$), multiple keys will collapse to the same bucket systematically. We need $\gcd(a,m)=1$.
    \item \textbf{Universal Bound:} The gold standard is $\Prob[\text{Collision}] \le 1/n$.
\end{itemize}

\section{Question 1}

\section{Optional Theory Problem}

\textbf{Question}: Recall that a set $H$ of hash functions (mapping the elements of a universe $U$ to the buckets $\{0, 1, 2, \ldots, n-1\}$) is universal if for every distinct $x, y \in U$, the probability $Prob[h(x) = h(y)]$ that $x$ and $y$ collide, assuming that the hash function $h$ is chosen uniformly at random from $H$, is at most $1/n$. In this problem you will prove that a collision probability of $1/n$ is essentially the best possible. Precisely, suppose that $H$ is a family of hash functions mapping $U$ to $\{0, 1, 2, \ldots, n-1\}$, as above. Show that there must be a pair $x, y \in U$ of distinct elements such that, if $h$ is chosen uniformly at random from $H$, then $Prob[h(x) = h(y)] \geq \frac{1}{n} - \frac{1}{|U|}$.

\smallskip

\textbf{Solution:} Let $H$ be a family of hash functions mapping a universe $U$ of size $M = |U|$ to $n$ buckets $\{0, 1, \dots, n-1\}$.
We wish to show that there exists at least one pair of distinct keys $x, y \in U$ such that:
\begin{equation}
    \Pr_{h \in H} [h(x) = h(y)] \ge \frac{1}{n} - \frac{1}{|U|}
\end{equation}

\subsection{Proof via the Average Argument}
We will prove this by showing that the \textbf{average} collision probability over \textit{all} possible pairs of keys is at least $\frac{1}{n} - \frac{1}{M}$. If the average is this high, at least one specific pair must meet or exceed this value.

\subsection{Step 1: Counting Pairs}
Let $N_{\text{pairs}}$ be the total number of distinct pairs of keys in the universe $U$.
\begin{equation}
    N_{\text{pairs}} = \binom{M}{2} = \frac{M(M-1)}{2}
\end{equation}

\subsection{Step 2: Analysis of a Single Hash Function}
Consider a fixed hash function $h \in H$. Let $c_i$ be the number of keys mapping to bucket $i$ (where $\sum_{i=0}^{n-1} c_i = M$).
The number of colliding pairs produced by this specific function, denoted $C(h)$, is the sum of the pairs within each bucket:
\begin{equation}
    C(h) = \sum_{i=0}^{n-1} \binom{c_i}{2} = \sum_{i=0}^{n-1} \frac{c_i(c_i - 1)}{2}
\end{equation}

To find the lower bound, we minimize $C(h)$. Since $f(x) = x(x-1)$ is a convex function, the sum is minimized by Jensen's Inequality when the keys are distributed as evenly as possible among the buckets (i.e., $c_i = M/n$ for all $i$).

The minimum number of collisions $C_{\text{min}}$ for \textit{any} hash function is:
\begin{align*}
    C_{\text{min}} &= \sum_{i=0}^{n-1} \frac{\frac{M}{n}(\frac{M}{n} - 1)}{2} \\
    &= n \cdot \frac{\frac{M}{n}(\frac{M - n}{n})}{2} \\
    &= \frac{M(M-n)}{2n}
\end{align*}

\subsection{Step 3: Calculating the Average Probability}
The average collision probability, $P_{\text{avg}}$, is the total minimum collisions divided by the total number of pairs.
\begin{align*}
    P_{\text{avg}} &= \frac{C_{\text{min}}}{N_{\text{pairs}}} \\
    &= \frac{ \frac{M(M-n)}{2n} }{ \frac{M(M-1)}{2} } \\
    &= \frac{M(M-n)}{2n} \cdot \frac{2}{M(M-1)} \\
    &= \frac{M-n}{n(M-1)}
\end{align*}

\subsection{Step 4: Comparison with the Bound}
We compare our calculated $P_{\text{avg}}$ with the target bound $\frac{1}{n} - \frac{1}{M}$.
First, simplify the target bound:
\begin{equation}
    \text{Target} = \frac{1}{n} - \frac{1}{M} = \frac{M - n}{nM}
\end{equation}

Now, compare the denominators of $P_{\text{avg}}$ and the Target:
\begin{equation}
    P_{\text{avg}} = \frac{M-n}{n(M-1)} \quad \text{vs} \quad \text{Target} = \frac{M-n}{nM}
\end{equation}
Since $M-1 < M$, the denominator of $P_{\text{avg}}$ is smaller, making the fraction larger. Therefore:
\begin{equation}
    \frac{M-n}{n(M-1)} > \frac{M-n}{nM}
\end{equation}
\begin{equation}
    P_{\text{avg}} > \frac{1}{n} - \frac{1}{|U|}
\end{equation}

\subsection{Conclusion}
Since the average collision probability over all pairs is strictly greater than $\frac{1}{n} - \frac{1}{|U|}$, there must exist at least one pair $(x, y)$ such that $\Pr[h(x)=h(y)] \ge P_{\text{avg}} > \frac{1}{n} - \frac{1}{|U|}$.

\end{document}